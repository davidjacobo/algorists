\documentclass[article]{beamer}
\usetheme{Warsaw}
\setbeamertemplate{footline}[frame number]

\usefonttheme[]{serif}
\usepackage{amsmath, latexsym, color, graphicx, amssymb, bm, here}
\usepackage{epsf, epsfig, pifont,tikz,subfigure}
\usepackage{graphics, calrsfs}
\usepackage{times}
\usepackage{fancybox,calc}
\usepackage{palatino,mathpazo}
\usepackage{amsfonts}
\usepackage{sidecap}
\usepackage{listings}
\usepackage{hyperref}

\title{Problem Solving Paradigms}
\author{David Jacobo \\ \href{mailto:jguillen@cimat.mx}{jguillen@cimat.mx}}
\date{\scriptsize{\today}}

\AtBeginSection[]
{
  \begin{frame}{Outline}
    \tableofcontents[currentsection]
  \end{frame}
}

\begin{document}

%%%%%%%%%%%%%Frame 1
\maketitle

%%%%%%%%%%%%%frame 2
\begin{frame}
\begin{block}{Paradigm:}
	Comes from Greek παράδειγμα (paradeigma), "pattern, example, sample"
\end{block}

\vspace{12mm}

\begin{flushright} {\footnotesize
	"If all you have is a hammer, everything looks like a nail" \newline Abraham Maslow, 1962.
} \end{flushright} 
\end{frame}

%%%%%%%%%%%%%%Frame 3
\section{Complete Search}
\subsection{Idea}

%%%%%%Frame 4
\begin{frame}[fragile]
\begin{block}{Complete search}
Also called backtracking or brute force, tries all the possible combinations within the search space. A bug-free implementation ensures you will always get the correct answer! 
\end{block}
\end{frame}

%%%%%%%Frame 5
\begin{frame}[fragile]
\begin{columns}
\column{.5\textwidth}
Pros
\begin{itemize}
	\item Easy to code.
	\item Always gives the correct answer.
	\item No complex data structures needed.
	\item Should be your first option.
\end{itemize}

\column{.5\textwidth}
Cons
\begin{itemize}
	\item Pretty bad complexity.
	\item Cannot handle large inputs.
	\item People tend to think your ki is low.
\end{itemize}
\end{columns}
\end{frame}

%%%%%%%%%%%%%%%%%%%%%%%
\subsection{Problem 1.- Coin change}

\begin{frame}[fragile]
	\frametitle{Coin change}
	{\footnotesize
	Having a known subset of coins (i.e. 1,3,4,10) and an infinite amount of each of those,
	determine what is the minimum amount of coins used to form N. 
	
	{\tiny
	\begin{lstlisting}
  6 int coins[] = {1,3,4,10};
  7 
  8 int get_coins_complete_search(int N){
  9     int min = MAX_N;
 10     
 11     for(int w=0;w < MAX_N;++w){
 12         for(int x=0;x < MAX_N;++x){
 13             for(int y=0;y < MAX_N;++y){
 14                 for(int z=0;z < MAX_N;++z){
 15                     if(w*coins[0] + x*coins[1] + y*coins[2] + z*coins[3] == N \
 16                         && w + x + y + z < min){
 17                         min = w + x + y + z;
 18                     }
 19                 }
 20             }
 21         }
 22     }
 23     
 24     return min;
 25 }
	\end{lstlisting}
	}
	}
\end{frame}

%%%%%%%Frame 6
\subsection{Problem 2.- E.T. calls 'homie'}

\begin{frame}[fragile]
	\frametitle{E.T. calls 'homie' 1/2}
	{\footnotesize
	Despite E.T. being part of a super advanced alien race... he has
	serious troubles to call his friends to pick him up when drunk. 
	\newline
	
	He can remember that his friend telephone number has 7 digits. 
	As he starts getting lucid again, he remembers the sum of the digits in the number is 42. 
	\newline
	
	You have to help E.T. printing a list of all the possible numbers for which
	both restrictions are valid (not because you like E.T., but he is at your home 
	and he is a real jackass in his current state). 
	}
\end{frame}

%%%%%%%%%%%%%% Frame 7
\begin{frame}[fragile]
	\frametitle{E.T. calls 'homie' 2/2}
	{\tiny
	\begin{lstlisting}
  8 void list_numbers(int state[],int k,int digits,int must_sum,int sum){
  9     if(k==digits){
 10         if(valid(must_sum, sum)){
 11             for(int i=1;i<=digits;i++){
 12                 cout<<state[i];
 13             }   
 14             cout<<endl;
 15         }   
 16     } else {
 17         ++k;
 18         int temp_sum;
 19 
 20         for(int i=0;i<10;++i){
 21             state[k] = i;
 22             temp_sum = sum + i;
 23             if(temp_sum > must_sum) continue;
 24             list_numbers(state, k, digits, must_sum, temp_sum);
 25         }   
 26     }
 27 }
	\end{lstlisting}
	}
	The complexity for this is $~O(n^{m})$ where n is the amount of different symbols [0-9] and m is the length of the telephone number.
\end{frame}

%%%%%%%%%%%%%%%%%%%%%%%%%%%%%%5
\section{Divide and conquer}
\subsection{Idea}

\begin{frame}[fragile]
	\begin{block}{Divide and conquer}
	Based on three steps: \newline
	1) Separate the problem in smaller cases \newline 
	2) Solve those smaller cases \newline
	3) Mix the solutions
	\end{block}
\end{frame}

%%%%%%%%%%%%%%%%%%%%%%%%%%%%%%Frame 10
\begin{frame}[fragile]
	\begin{columns}
		\column{.5\textwidth}
		Pros
		\begin{itemize}
			\item Easy to code as a recursion.
			\item A lot of efficient algorithms
				are written this way!
			\item Search space is reduced by half every time.
		\end{itemize}
		\column{.5\textwidth}
		Cons
		\begin{itemize}
			\item You need to be careful about 
				base cases (infinite loop).
		\end{itemize}
	\end{columns}
\end{frame}

%%%%%%%%%%%%%%%%%%%%%%%%%%%% Frame 11
\subsection{Problem 3.- RSQ}

\begin{frame}[fragile]
	\frametitle{Problem 3.- Range Sum Query 1/3}
	{\footnotesize
		Given an static array of N values A = $[2,3,5,1,3,7,3,8]$ you are asked to
		respond to queries like "What is the sum from i to j?", where i and j
		are between [0,N-1] and $i<=j$. \newline
	}
\end{frame}

%%%%%%%%%%%%%%%%%%%%%%%%%%%%%

\begin{frame}[fragile]
	\frametitle{Problem 3.- Range Sum Query 2/3}
	In order to answer the queries it is necessary to build a segment tree first:	
	{\tiny
		\begin{lstlisting}
  8 int build_tree(int ind,int p,int q){
  9     if(p==q){
 10         return st[ind] = A[p];
 11     } else {
 12         int m = (p+q)>>1;
 13         int left = ind<<1;
 14         int right = left+1;
 15 
 16         int sum = build_tree(left  ,p  ,m);
 17         sum +=    build_tree(right ,m+1,q);
 18 
 19         return st[ind] = sum;
 20     }
 21 }	
		\end{lstlisting}
	}	
		$O(n log_{2} n)$ is a great building complexity, but how do we make a query? ...
\end{frame}

%%%%%%%%%%%%%%%%%%%%%%%%%%%%%%
\begin{frame}[fragile]
	\frametitle{Problem 3.- Range Sum Query 3/3}
	The query code could look like this:
	{\tiny
		\begin{lstlisting}
 23 int query(int ind,int p,int q,int i,int j){
 24     if(p==i && q==j) return st[ind];
 25     
 26     int m = (p+q)>>1;
 27     int left = ind<<1;
 28     int right = left+1;
 29 
 30     if(i> m) {
 31         return query(right,  m+1, q, i, j);
 32     } else if(j<= m) {
 33         return query(left ,    p, m, i, j);
 34     } else {
 35         return query(right,  m+1, q, m+1, j)\
 36              + query(left ,    p, m,   i, m);
 37     }        
 38 }  
		\end{lstlisting}
	}
	$O(log_{2} n)$ per query, isn't that cool enough? Well this data structure supports $O(log_{2} n)$ updates, for the case where A is a dynamic array.
\end{frame}
	
%%%%%%%%%%%%%%%%%%%%%%%%%%%%%%%
\section{Dynamic Programming}
\subsection{Idea}

\begin{frame}[fragile]
	\begin{block}{Dynamic Programming}
	Paradigm focused on constructing the answer of a problem by
	solving the smaller instances first. Some people just call it memoization. 
	\\
	A problem needs to exhibit two properties to be approached this way: \textbf{optimal substructure and overlapping sub-problems}.
	\end{block}
\end{frame}
%%%%%%%%%%%%%%%%%%%%%%%%%%%%%%Frame 11

\begin{frame}[fragile]
	\begin{columns}
		\column{.5\textwidth}
		Pros
		\begin{itemize}
			\item Efficient, never computes the same case twice.
			\item Correct, it evaluates all the possible solutions.
			\item 2 flavours!, bottom-up and top-down.
		\end{itemize}
		\column{.5\textwidth}
		Cons
		\begin{itemize}
			\item Memory costly, is it feasible to store the answers for the N-1 sub-problems?
		\end{itemize}
	\end{columns}
\end{frame}

%%%%%%%%%%%%%%%%%%%%%%%%%%%%%%%%%%%%%
\subsection{Recurrence formulas}
\begin{frame}[fragile]
	DP(Dynamic programming) based problems are frequently described in terms of a recurrence formula, the one for our coin change problem can be
	written as:
	\vspace{8mm}
	
	\begin{equation}
  	f(N)=\begin{cases}
  	0, & \text{if $N==0$}.\\
    1 + min(f(N-coin_{i})) \forall coin_{i} \in \textbf{S}, & \text{otherwise}.
 	 \end{cases}
	\end{equation} 
	\begin{flushright}
	{\footnotesize  where \textbf{S} is a set containing the different coins.}
	\end{flushright}
	\vspace{5mm}
	
	
	\textbf{Base cases} are those for which you already know the answer, for this scenario, we know we need 0 coins to give 0 pesos back.
	But it has a potential problem. What would happen if N is negative?
\end{frame}

%%%%%%%%%%%%%%%%%%%%%%%%%%%%%%%%%%%%%
\subsection{Problem 4.- Coin change with bottom up}
\begin{frame}[fragile]
	This version uses 1 coin at a time, it builds the optimal solution for the
	[0, N-1] cases before calculating N. 
	
	{\tiny
		\begin{lstlisting}
 27 int get_coins_dp_bottom_up(int N){
 28     int  memo[N+1];
 29     int coin_types = sizeof(coins) / sizeof(int);
 30 
 31     for(int i=1;i <= N;++i) memo[i] = INF;
 32     memo[0] = 0;
 33 
 34     for(int i=0;i < coin_types;++i){
 35         for(int j=1;j <= N;++j){
 36             if(j>=coins[i]) {
 37                 memo[j] = min(memo[j] , memo[j-coins[i]]+1);
 38             }
 39         }
 40     }
 41 
 42     return memo[N];
 43 }
		\end{lstlisting}
	}
	
	Being N the amount of change to give, and M the different types of coins, $O(MN)$ describes the running time complexity and $O(N)$ the space complexity.
\end{frame}

%%%%%%%%%%%%%%%%%%%%

\subsection{Problem 5.- Coin change with top down}
\begin{frame}[fragile]
	This version perfectly reflects the formula we described before, so 
	its easier to implement if you are comfortable with recurrences.
	{\tiny
		\begin{lstlisting}
 45 int top_down(int N,int *memo,int coin_types){
 46     if(N<0) return INF;
 47     if(INF!=memo[N]) return memo[N];
 48 
 49     int min = INF;
 50     for(int i=0;i < coin_types;++i){
 51         int temp = top_down(N-coins[i], memo, coin_types)+1;
 52         if(temp < min)
 53             min = temp;
 54     }
 55 
 56    return memo[N] = min;
 57 }
 58 
 59 int get_coins_dp_top_down(int N){
 60     int coin_types = sizeof(coins) / sizeof(int);
 61     int memo[N+1];
 62 
 63     for(int i=0;i <= N;++i)
 64         memo[i] = INF;
 65     memo[0] = 0;
 66 
 67     top_down(N, memo, coin_types);
 68     return memo[N];
 69 } 
		\end{lstlisting}
	}
\end{frame}

%%%%%%%%%%%%%%%%%%%%%%
\section{Greedy algorithms}
\subsection{Idea}

\begin{frame}[fragile]
	\begin{block}{Greedy algorithms}
	Technique based on making the best local decision every time expecting to
	obtain the optimal solution. A greedy algorithm need to exhibit an \textbf{optimal substructure} too. 
	\end{block}
\end{frame}

%%%%%%%%%%%%%%%%%%%%%%%%% Frame 12
\begin{frame}[fragile]
	\begin{columns}
		\column{.5\textwidth}
		Pros
		\begin{itemize}
			\item Faster than DP.
			\item Optimal, if the problem exhibits an optimal substructure.
			\item Easier code most of the times.
		\end{itemize}
		\column{.5\textwidth}
		Cons
		\begin{itemize}
			\item Not easy to prove (correctness proof).
			\item Heuristics are needed sometimes.
		\end{itemize}
	\end{columns}
\end{frame}

%%%%%%%%%%%%%%%%%%%%%%%%%%%%%5
\subsection{Problem 6.- Coin change with greedy}
\begin{frame}[fragile]
	The code for the coin change problem could look like this:
	{\tiny
		\begin{lstlisting}
 71 int get_coins_greedy(int N){
 72     int coins_counter = 0;
 73     int coin_types = sizeof(coins) / sizeof(int);
 74 
 75     for(int i=--coin_types;i >= 0;--i){
 76         while(N >= coins[i]){
 77             ++coins_counter;
 78             N-= coins[i];
 79         }
 80     }
 81 
 82     return coins_counter;
 83 }
		\end{lstlisting}
	}
	
	As silly as it looks, this code runs in $O(N)$ and needs no extra saving space, but, does it really works for any coins set? 
\end{frame}

%%%%%%%%%%%%%%%%%%%%%%%%%%%%%%Frame 22
\begin{frame}[plain]
\frametitle{}
\begin{center}
\Huge{\color{blue}{Q \& A}}
\end{center}
\end{frame}

%%%%%%%%%%%%%%%%%%%%%%%%%%%%%%%%
\begin{frame}[plain]
	\textbf{References}
	\begin{itemize}
		\item \href{https://sites.google.com/site/stevenhalim/}{Competitive Programming site}
		\item \href{https://github.com/davidjacobo/algorists/}{Algorists' repository}
	\end{itemize}
\end{frame}
\end{document}