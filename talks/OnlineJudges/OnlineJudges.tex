%%%%%%%%%%%%%%%%%%%%%%%%%%%%%%%%%%%%%%%%%
% Beamer Presentation
% LaTeX Template
% Version 1.0 (10/11/12)
%
% This template has been downloaded from:
% http://www.LaTeXTemplates.com
%
% License:
% CC BY-NC-SA 3.0 (http://creativecommons.org/licenses/by-nc-sa/3.0/)
%
%%%%%%%%%%%%%%%%%%%%%%%%%%%%%%%%%%%%%%%%%

%----------------------------------------------------------------------------------------
%	PACKAGES AND THEMES
%----------------------------------------------------------------------------------------

\documentclass{beamer}

\mode<presentation> {

% The Beamer class comes with a number of default slide themes
% which change the colors and layouts of slides. Below this is a list
% of all the themes, uncomment each in turn to see what they look like.

%\usetheme{default}
%\usetheme{AnnArbor}
%\usetheme{Antibes}
%\usetheme{Bergen}
%\usetheme{Berkeley}
%\usetheme{Berlin}
%\usetheme{Boadilla}
%\usetheme{CambridgeUS}
%\usetheme{Copenhagen}
%\usetheme{Darmstadt}
%%\usetheme{Dresden}
%\usetheme{Frankfurt}
%\usetheme{Goettingen}
%\usetheme{Hannover}
%\usetheme{Ilmenau}
%\usetheme{JuanLesPins}
%\usetheme{Luebeck}
\usetheme{Madrid}
%\usetheme{Malmoe}
%\usetheme{Marburg}
%\usetheme{Montpellier}
%\usetheme{PaloAlto}
%\usetheme{Pittsburgh}
%\usetheme{Rochester}
%\usetheme{Singapore}
%\usetheme{Szeged}
%\usetheme{Warsaw}

% As well as themes, the Beamer class has a number of color themes
% for any slide theme. Uncomment each of these in turn to see how it
% changes the colors of your current slide theme.

%\usecolortheme{albatross}
%\usecolortheme{beaver}
%\usecolortheme{beetle}
%\usecolortheme{crane}
%\usecolortheme{dolphin}
%%\usecolortheme{dove}
%\usecolortheme{fly}
%\usecolortheme{lily}
%\usecolortheme{orchid}
\usecolortheme{rose}
%%\usecolortheme{seagull}
%%%\usecolortheme{seahorse}
%\usecolortheme{whale}
%%\usecolortheme{wolverine}

%\setbeamertemplate{footline} % To remove the footer line in all slides uncomment this line
%\setbeamertemplate{footline}[page number] % To replace the footer line in all slides with a simple slide count uncomment this line

%\setbeamertemplate{navigation symbols}{} % To remove the navigation symbols from the bottom of all slides uncomment this line
}

\usepackage{graphicx} % Allows including images
\usepackage{ragged2e} % Jusitfy
\usepackage{booktabs} % Allows the use of \toprule, \midrule and \bottomrule in tables
\usepackage{lmodern}
\usepackage{amsfonts}
\usepackage{amsmath}
\usepackage{hyperref}
\usepackage{mathtools}
\usepackage{listings} % C++ code


% -------------
%   CONFIG
% -------------
\lstset{language=C++,
                basicstyle=\footnotesize\ttfamily,
                keywordstyle=\footnotesize\color{blue}\ttfamily,
}
\addtobeamertemplate{block begin}{}{\justifying} %Justify
\hypersetup{
  colorlinks, linkcolor=green
}
%----------------------------


%----------------------------------------------------------------------------------------
%	TITLE PAGE
%----------------------------------------------------------------------------------------

\title[Online Judges]{Online Judges} % The short title appears at the bottom of every slide, the full title is only on the title page

\author{Ulises M\'endez Mart\'{i}nez} % Your name

\institute[UTM] % Your institution as it will appear on the bottom of every slide, may be shorthand to save space
{
Algorist Weekly Talks \\ % Your institution for the title page
\medskip
\textit{ulisesmdzmtz@gmail.com} % Your email address
}
\date{\today} % Date, can be changed to a custom date

\begin{document}

\begin{frame}
\titlepage % Print the title page as the first slide
\end{frame}

%----------------------------------------------------------------------------------------
%	PRESENTATION SLIDES
%----------------------------------------------------------------------------------------

%--------------------
%      LOGOS
%--------------------
\begin{frame}
	\frametitle{Online Judges}
	\begin{figure}[!htb]
	 \minipage{0.23\textwidth}
	  \includegraphics[width=\linewidth]{images/judges/usaco}%
     \endminipage\hfill
     \minipage{0.23\textwidth}
      \includegraphics[width=\linewidth]{images/judges/uva}%
     \endminipage\hfill
     \minipage{0.23\textwidth}
      \includegraphics[width=\linewidth]{images/judges/spoj}%
     \endminipage\hfill
     \minipage{0.23\textwidth}
      \includegraphics[width=\linewidth]{images/judges/projecteuler}%
     \endminipage\hfill
	\end{figure}
	\begin{figure}[!htb]
 	 \minipage{0.23\textwidth}
      \includegraphics[width=\linewidth]{images/judges/hackerrank}%
     \endminipage\hfill
     \minipage{0.23\textwidth}
      \includegraphics[width=\linewidth]{images/judges/codeforces}%
     \endminipage\hfill
     \minipage{0.23\textwidth}
      \includegraphics[width=\linewidth]{images/judges/codechef}%
     \endminipage\hfill
     \minipage{0.23\textwidth}
      \includegraphics[width=\linewidth]{images/judges/topcoder}%
     \endminipage\hfill
	\end{figure}
    \begin{figure}[!htb]
     \minipage{0.23\textwidth}
      \includegraphics[width=\linewidth]{images/judges/coj}%
     \endminipage\hfill
     \minipage{0.23\textwidth}
      \includegraphics[width=\linewidth]{images/judges/hackerearth}%
     \endminipage\hfill
     \minipage{0.23\textwidth}
      \includegraphics[width=\linewidth]{images/judges/omegaup}%
     \endminipage\hfill
     \minipage{0.23\textwidth}
      \includegraphics[width=\linewidth]{images/judges/livearchive}%
     \endminipage\hfill
	\end{figure}
\end{frame}

%--------------------
%      INTRO
%--------------------
\begin{frame}
	\frametitle{Competitive programming}
		\begin{block}{From wikipedia}
		A programming competition generally involves the host presenting a set of logical or 		mathematical problems to the contestants, and contestants are required to write computer programs capable of solving each problem. \newline \\
		\textbf{Judging is based mostly upon number of problems solved and time spent for 			writing successful solutions, but may also include other factors (quality of output 			produced, execution time, program size, etc.)}
		\end{block}
\end{frame}

%---------------------------
%      VEREDICTS
%---------------------------
\begin{frame}
	\frametitle {Common Veredicts I}
	\begin{block}{Submissions}
	Your program will be compiled and run in our system, and the automatic judge will test 		it with some inputs and outputs, or perhaps with a specific judge tool. After some 			seconds	or minutes, you'll receive by e-mail (or you'll see in the web) one of these 			answers:
	\end{block}
	
	\begin{block}{Compile Error (CE):}
	The compiler could not compile your program. Of course, warning messages are not error messages. The compiler output messages are reported you by e-mail.
	\end{block}
	
	\begin{block}{Accepted (AC):}
	OK! Your program is correct! It produced the right answer in reasoneable time and within the limit memory usage. Congratulations!
	\end{block}
	
	

\end{frame}

\begin{frame}
	\frametitle {Common Veredicts II}
	
	\begin{block}{Wrong Answer (WA):}
	Correct solution not reached for the inputs. 
	%The inputs and outputs that we use to test the programs are not public so you'll have to spot the bug by yourself (it is recomendable to get accustomed to a true contest dynamic ;-)). 
	If you truly think your code is correct, you can contact us using the link on the left. Judge's outputs are not always correct...
	\end{block}
	
	\begin{block}{Runtime Error (RE):}
	Your program failed during the execution (segmentation fault, floating point exception...). The exact cause is not reported to the user to avoid hacking. Be sure that your program returns a 0 code to the shell. If you're using Java, please follow all the submission specifications.
	\end{block}
	
	\begin{block}{Time Limit Exceeded (TL):}
	Your program tried to run during too much time; this error doesn't allow you to know if your program would reach the correct solution to the problem or not.
	\end{block}
	
\end{frame}
	
%\begin{frame}
%	\frametitle {Common Veredicts III}
	
%	\begin{block}{Presentation Error (PE):}
%	Your program outputs are correct but are not presented in the correct way. Check for spaces, justify, line feeds...
%	\end{block}
	
%	\begin{block}{In Queue (QU):}
%	The judge is busy and can't attend your submission. It will be judged as soon as possible.
%	\end{block}
	
%	\begin{block}{Memory Limit Exceeded (ML):}
%	Your program tried to use more memory than the judge allows. If you are sure that such problem needs more memory, please contact us.
%	\end{block}	
	
%\end{frame}

%\begin{frame}
%	\frametitle {Common Veredicts IV}
%	\begin{block}{Output Limit Exceeded (OL):}
%	Your program tried to write too much information. This usually occurs if it goes into a infinite loop.
%	\end{block}
%	
%	\begin{block}{Restricted Function (RF):}
%	Your program is trying to use a function that we considered harmful to the system. If you get this verdict you probably know why...
%	\end{block}
	
%	\begin{block}{Can't Be Judged (CJ):}
%	The judge doesn't have test input and outputs for the selected problem. While choosing a problem be careful to ensure that the judge will be able to judge it!
%	\end{block}

%	And so on...	
	
%\end{frame}


%------------------------------
%       COJ
%------------------------------

\begin{frame}
	\frametitle{Caribbean Online Judge}
	\begin{figure}[t]
		\href{http://coj.uci.cu/general/faqs.xhtml?lang=es}{\includegraphics[width=3cm,height=2.5cm,keepaspectratio]{images/judges/coj}}
	\end{figure}
	
	\begin{columns}[c] % The "c" option specifies centered vertical alignment while the "t" option is used for top vertical alignment
		\column{.45\textwidth} % Left column and width
		\textbf{Pros}
		\begin{enumerate}
			\item Some Spanish descriptions  
			\item Progressive contest
			\item Ranks \& Leaderboards
			\item Code Backup
			\item Dynamic score problems
			\item Frozen scoreboard
		\end{enumerate}
		\column{.5\textwidth} % Right column and width
		\textbf{Cons}
		\begin{enumerate}
			\item Most recycling problems
			\item Private contests
			\item Very strict with I/O
			\item Access
			\item Frozen scoreboard
		\end{enumerate}
	\end{columns}
	\begin{block}{Other resources}
		\begin{itemize}
			\item \href{https://coj-forum.uci.cu/viewtopic.php?f=35&t=2846}{Caribbean ICPC Information} 
		\end{itemize}
	\end{block}

\end{frame}

%--------------------
%      UVA OJ
%--------------------

\begin{frame}
	\frametitle{Universidad of Valladolid Online Judge}
	\begin{figure}[t]
		\href{https://uva.onlinejudge.org/}{\includegraphics[width=4cm,height=4cm,keepaspectratio]{images/judges/uva}}
	\end{figure}
	
	\begin{columns}[c] % The "c" option specifies centered vertical alignment while the "t" option is used for top vertical alignment
		\column{.45\textwidth} % Left column and width
		\textbf{Pros}
		\begin{enumerate}
			\item Vast quantity of problems 
			\item Many online references 
			\item Good Analityc's Tools 
			\item CP Book is based completely on it
		\end{enumerate}
		\column{.5\textwidth} % Right column and width
		\textbf{Cons}
		\begin{enumerate}
			\item Poor Online Connections
			\item No code backup
			\item No feedback
			\item Heavy traffic of users (delays in response)
		\end{enumerate}
	\end{columns}
	\begin{block}{Tools}
	\center
	\href{http://uhunt.felix-halim.net/} {\includegraphics[width=1cm,height=1cm,keepaspectratio]{images/tools/uvahunting}}
	\href{https://www.udebug.com/}	{\includegraphics[width=3cm,height=3cm,keepaspectratio]{images/tools/udebug}}
	\href{http://uvatoolkit.com/problemssolve.php}{\includegraphics[width=1cm,height=1cm,keepaspectratio]{images/tools/uvatoolkit}}
	\end{block}

\end{frame}

%----------------------------
%     CODECHEF  OJ
%----------------------------

\begin{frame}
	\frametitle{Codechef }
	\begin{figure}[t]
		\href{https://www.codechef.com/getting-started}{\includegraphics[width=3cm,height=3cm,keepaspectratio]{images/judges/codechef}}
	\end{figure}
	
	\begin{columns}[c] % The "c" option specifies centered vertical alignment while the "t" option is used for top vertical alignment
		\column{.45\textwidth} % Left column and width
		\textbf{Pros}
		\begin{enumerate}
			\item Active Contests Schedule  
			\item \href{https://blog.codechef.com/}{Community Blogs, Resources}
			\item Problems level clasification
			\item \href{https://www.codechef.com/ide}{Online Compiler}
			\item Sponsors \& Prizes
			\item Code Backup
		\end{enumerate}
		\column{.5\textwidth} % Right column and width
		\textbf{Cons}
		\begin{enumerate}
			\item Troubleshooting with large audiences
			\item Math problems tend to be specific
			\item Not unified rank
		\end{enumerate}
	\end{columns}
	\begin{block}{Contest}
	\center
	\href{https://www.codechef.com/JAN16} {\includegraphics[width=3cm,height=3cm,keepaspectratio]{images/contests/long_challenge}}
	\href{https://www.codechef.com/LTIME32}	{\includegraphics[width=3cm,height=3cm,keepaspectratio]{images/contests/lunch_time}}
	\href{https://www.codechef.com/COOK66}{\includegraphics[width=3cm,height=3cm,keepaspectratio]{images/contests/cook_off}}
	\end{block}

\end{frame}

%------------------------------
%       TOP CODER
%------------------------------

\begin{frame}
	\frametitle{Universidad of Valladolid Online Judge}
	\begin{figure}[t]
		\href{https://uva.onlinejudge.org/}{\includegraphics[width=4cm,height=4cm,keepaspectratio]{images/judges/uva}}
	\end{figure}
	
	\begin{columns}[c] % The "c" option specifies centered vertical alignment while the "t" option is used for top vertical alignment
		\column{.45\textwidth} % Left column and width
		\textbf{Pros}
		\begin{enumerate}
			\item Profitable 
			\item Class vs Complete Program 
			\item High Level Discussions / Problems
			\item Rank \& Prizes \& Achievements
		\end{enumerate}
		\column{.5\textwidth} % Right column and width
		\textbf{Cons}
		\begin{enumerate}
			\item Contest Platforms
			\item No code backup
			\item No feedback
			\item Heavy traffic of users (delays in response)
		\end{enumerate}
	\end{columns}
	\begin{block}{Tools}
	\center
	\href{http://uhunt.felix-halim.net/} {\includegraphics[width=1cm,height=1cm,keepaspectratio]{images/tools/uvahunting}}
	\href{https://www.udebug.com/}	{\includegraphics[width=3cm,height=3cm,keepaspectratio]{images/tools/udebug}}
	\href{http://uvatoolkit.com/problemssolve.php}{\includegraphics[width=1cm,height=1cm,keepaspectratio]{images/tools/uvatoolkit}}
	\end{block}

\end{frame}


%------------------------------
%       HACKERRANK
%------------------------------

\begin{frame}
	\frametitle{Universidad of Valladolid Online Judge}
	\begin{figure}[t]
		\href{https://uva.onlinejudge.org/}{\includegraphics[width=4cm,height=4cm,keepaspectratio]{images/judges/uva}}
	\end{figure}
	
	\begin{columns}[c] % The "c" option specifies centered vertical alignment while the "t" option is used for top vertical alignment
		\column{.45\textwidth} % Left column and width
		\textbf{Pros}
		\begin{enumerate}
			\item Vast quantity of problems 
			\item Many online references 
			\item Good Analityc's Tools 
			\item CP Book is based completely on it
		\end{enumerate}
		\column{.5\textwidth} % Right column and width
		\textbf{Cons}
		\begin{enumerate}
			\item Poor Online Connections
			\item No code backup
			\item No feedback
			\item Heavy traffic of users (delays in response)
		\end{enumerate}
	\end{columns}
	\begin{block}{Tools}
	\center
	\href{http://uhunt.felix-halim.net/} {\includegraphics[width=1cm,height=1cm,keepaspectratio]{images/tools/uvahunting}}
	\href{https://www.udebug.com/}	{\includegraphics[width=3cm,height=3cm,keepaspectratio]{images/tools/udebug}}
	\href{http://uvatoolkit.com/problemssolve.php}{\includegraphics[width=1cm,height=1cm,keepaspectratio]{images/tools/uvatoolkit}}
	\end{block}

\end{frame}


%------------------------------
%       CODEFORCES
%------------------------------

\begin{frame}
	\frametitle{Universidad of Valladolid Online Judge}
	\begin{figure}[t]
		\href{https://uva.onlinejudge.org/}{\includegraphics[width=4cm,height=4cm,keepaspectratio]{images/judges/uva}}
	\end{figure}
	
	\begin{columns}[c] % The "c" option specifies centered vertical alignment while the "t" option is used for top vertical alignment
		\column{.45\textwidth} % Left column and width
		\textbf{Pros}
		\begin{enumerate}
			\item Vast quantity of problems 
			\item Many online references 
			\item Good Analityc's Tools 
			\item CP Book is based completely on it
		\end{enumerate}
		\column{.5\textwidth} % Right column and width
		\textbf{Cons}
		\begin{enumerate}
			\item Poor Online Connections
			\item No code backup
			\item No feedback
			\item Heavy traffic of users (delays in response)
		\end{enumerate}
	\end{columns}
	\begin{block}{Tools}
	\center
	\href{http://uhunt.felix-halim.net/} {\includegraphics[width=1cm,height=1cm,keepaspectratio]{images/tools/uvahunting}}
	\href{https://www.udebug.com/}	{\includegraphics[width=3cm,height=3cm,keepaspectratio]{images/tools/udebug}}
	\href{http://uvatoolkit.com/problemssolve.php}{\includegraphics[width=1cm,height=1cm,keepaspectratio]{images/tools/uvatoolkit}}
	\end{block}

\end{frame}



%\begin{frame}
%\frametitle{Overview} % Table of contents slide, comment this block out to remove it
%\tableofcontents % Throughout your presentation, if you choose to use 
%\end{frame}

%------------------------------------------------

%------------------------------------------------
%\begin{frame}
%\frametitle{Figure}
%Uncomment the code on this slide to include your own image from the same directory as the template .TeX file.
%\begin{figure}
%\includegraphics[width=0.4\linewidth]{z_image.png}
%\end{figure}
%\end{frame}

%------------------------------------------------

%\begin{frame}[fragile] % Need to use the fragile option when verbatim is used in the slide
%\frametitle{Citation}
%An example of the \verb|\cite| command to cite within the presentation:\\~

%This statement requires citation \cite{p1}.
%\end{frame}

%------------------------------------------------

%\begin{frame}
%\frametitle{References}
%\footnotesize{
%\begin{thebibliography}{99} % Beamer does not support BibTeX so references must be inserted manually as below
%\bibitem[Smith, 2012]{p1} John Smith (2012)
%\newblock Title of the publication
%\newblock \emph{Journal Name} 12(3), 45 -- 678.
%\end{thebibliography}
%}
%\end{frame}

%\begin{frame}
%\frametitle{Theorem}
%\begin{theorem}[Mass--energy equivalence]
%$E = mc^2$
%\end{theorem}
%\end{frame}
%------------------------------------------------
%\begin{frame}
%\frametitle{Bullet Points}
%\begin{itemize}
%\item Lorem ipsum 45 \(45\) \textbf{45} \(\textbf{45}\) dolor sit amet, consectetur adipiscing elit
%\item Aliquam blandit faucibus nisi, sit amet dapibus enim tempus eu
%\item Nulla commodo, erat quis gravida posuere, elit lacus lobortis est, quis porttitor odio mauris at libero
%\item Nam cursus est eget velit posuere pellentesque
%\item Vestibulum faucibus velit a augue condimentum quis convallis nulla gravida
%\end{itemize}
%\end{frame}


%\begin{frame}
%\frametitle{Multiple Columns}
%\begin{columns}[c] % The "c" option specifies centered vertical alignment while the "t" option is used for top vertical alignment

%\column{.45\textwidth} % Left column and width
%\textbf{Heading}
%\begin{enumerate}
%\item Statement
%\item Explanation
%\item Example
%\end{enumerate}

%\column{.5\textwidth} % Right column and width
%Lorem ipsum dolor sit amet, consectetur adipiscing elit. Integer lectus nisl, ultricies in feugiat rutrum, porttitor sit amet augue. Aliquam ut tortor mauris. Sed volutpat ante purus, quis accumsan dolor.

%\end{columns}
%\end{frame}

%----------------------------------------------------------------------------------------

\end{document} 
