%%%%%%%%%%%%%%%%%%%%%%%%%%%%%%%%%%%%%%%%%
% Beamer Presentation
% LaTeX Template
% Version 1.0 (10/11/12)
%
% This template has been downloaded from:
% http://www.LaTeXTemplates.com
%
% License:
% CC BY-NC-SA 3.0 (http://creativecommons.org/licenses/by-nc-sa/3.0/)
%
%%%%%%%%%%%%%%%%%%%%%%%%%%%%%%%%%%%%%%%%%

%----------------------------------------------------------------------------------------
%	PACKAGES AND THEMES
%----------------------------------------------------------------------------------------

\documentclass{beamer}

\mode<presentation> {

% The Beamer class comes with a number of default slide themes
% which change the colors and layouts of slides. Below this is a list
% of all the themes, uncomment each in turn to see what they look like.

%\usetheme{default}
%\usetheme{AnnArbor}
%\usetheme{Antibes}
%\usetheme{Bergen}
%\usetheme{Berkeley}
%\usetheme{Berlin}
%\usetheme{Boadilla}
%\usetheme{CambridgeUS}
%\usetheme{Copenhagen}
%\usetheme{Darmstadt}
%%\usetheme{Dresden}
%\usetheme{Frankfurt}
%\usetheme{Goettingen}
%\usetheme{Hannover}
%\usetheme{Ilmenau}
%\usetheme{JuanLesPins}
%\usetheme{Luebeck}
\usetheme{Madrid}
%\usetheme{Malmoe}
%\usetheme{Marburg}
%\usetheme{Montpellier}
%\usetheme{PaloAlto}
%\usetheme{Pittsburgh}
%\usetheme{Rochester}
%\usetheme{Singapore}
%\usetheme{Szeged}
%\usetheme{Warsaw}

% As well as themes, the Beamer class has a number of color themes
% for any slide theme. Uncomment each of these in turn to see how it
% changes the colors of your current slide theme.

%\usecolortheme{albatross}
\usecolortheme{beaver}
%\usecolortheme{beetle}
%\usecolortheme{crane}
%\usecolortheme{dolphin}
%%\usecolortheme{dove}
%\usecolortheme{fly}
%\usecolortheme{lily}
%\usecolortheme{orchid}
%\usecolortheme{rose}
%\usecolortheme{seagull}
%\usecolortheme{seahorse}
%\usecolortheme{whale}
%\usecolortheme{wolverine}

%\setbeamertemplate{footline} % To remove the footer line in all slides uncomment this line
%\setbeamertemplate{footline}[page number] % To replace the footer line in all slides with a simple slide count uncomment this line

%\setbeamertemplate{navigation symbols}{} % To remove the navigation symbols from the bottom of all slides uncomment this line
}

\usepackage{graphicx} % Allows including images
\usepackage{booktabs} % Allows the use of \toprule, \midrule and \bottomrule in tables
\usepackage{lmodern}
\usepackage{amsfonts}
\usepackage{amsmath}
\usepackage{mathtools}
\usepackage{listings} % C++ code
\lstset{language=C++,
                basicstyle=\footnotesize\ttfamily,
                keywordstyle=\footnotesize\color{blue}\ttfamily,
}
%----------------------------------------------------------------------------------------
%	TITLE PAGE
%----------------------------------------------------------------------------------------

\title[Online Judges]{Online Judges} % The short title appears at the bottom of every slide, the full title is only on the title page

\author{Ulises M\'endez Mart\'{i}nez} % Your name

\institute[UTM] % Your institution as it will appear on the bottom of every slide, may be shorthand to save space
{
Algorist Weekly Talks \\ % Your institution for the title page
\medskip
\textit{ulisesmdzmtz@gmail.com} % Your email address
}
\date{\today} % Date, can be changed to a custom date

\begin{document}

\begin{frame}
\titlepage % Print the title page as the first slide
\end{frame}

%----------------------------------------------------------------------------------------
%	PRESENTATION SLIDES
%----------------------------------------------------------------------------------------

%--------------------
%      LOGOS
%--------------------
\begin{frame}
	\frametitle{Online Judges}
	\begin{figure}[!htb]
	 \minipage{0.23\textwidth}
	  \includegraphics[width=\linewidth]{images/judges/usaco}~%
     \endminipage\hfill
     \minipage{0.23\textwidth}
      \includegraphics[width=\linewidth]{images/judges/uva}~%
     \endminipage\hfill
     \minipage{0.23\textwidth}
      \includegraphics[width=\linewidth]{images/judges/spoj}%
     \endminipage\hfill
     \minipage{0.23\textwidth}
      \includegraphics[width=\linewidth]{images/judges/projecteuler}%
     \endminipage\hfill
	\end{figure}
	\begin{figure}[!htb]
 	 \minipage{0.23\textwidth}
      \includegraphics[width=\linewidth]{images/judges/hackerrank}%
     \endminipage\hfill
     \minipage{0.23\textwidth}
      \includegraphics[width=\linewidth]{images/judges/codeforces}%
     \endminipage\hfill
     \minipage{0.23\textwidth}
      \includegraphics[width=\linewidth]{images/judges/codechef}%
     \endminipage\hfill
     \minipage{0.23\textwidth}
      \includegraphics[width=\linewidth]{images/judges/topcoder}%
     \endminipage\hfill
	\end{figure}
    \begin{figure}[!htb]
     \minipage{0.23\textwidth}
      \includegraphics[width=\linewidth]{images/judges/coj}%
     \endminipage\hfill
     \minipage{0.23\textwidth}
      \includegraphics[width=\linewidth]{images/judges/hackerearth}%
     \endminipage\hfill
     \minipage{0.23\textwidth}
      \includegraphics[width=\linewidth]{images/judges/omegaup}%
     \endminipage\hfill
     \minipage{0.23\textwidth}
      \includegraphics[width=\linewidth]{images/judges/livearchive}%
     \endminipage\hfill
	\end{figure}
\end{frame}

%\begin{frame}
%\frametitle{Overview} % Table of contents slide, comment this block out to remove it
%\tableofcontents % Throughout your presentation, if you choose to use 
%\end{frame}

%------------------------------------------------

%------------------------------------------------
%\begin{frame}
%\frametitle{Figure}
%Uncomment the code on this slide to include your own image from the same directory as the template .TeX file.
%\begin{figure}
%\includegraphics[width=0.4\linewidth]{z_image.png}
%\end{figure}
%\end{frame}

%------------------------------------------------

%\begin{frame}[fragile] % Need to use the fragile option when verbatim is used in the slide
%\frametitle{Citation}
%An example of the \verb|\cite| command to cite within the presentation:\\~

%This statement requires citation \cite{p1}.
%\end{frame}

%------------------------------------------------

%\begin{frame}
%\frametitle{References}
%\footnotesize{
%\begin{thebibliography}{99} % Beamer does not support BibTeX so references must be inserted manually as below
%\bibitem[Smith, 2012]{p1} John Smith (2012)
%\newblock Title of the publication
%\newblock \emph{Journal Name} 12(3), 45 -- 678.
%\end{thebibliography}
%}
%\end{frame}

%\begin{frame}
%\frametitle{Theorem}
%\begin{theorem}[Mass--energy equivalence]
%$E = mc^2$
%\end{theorem}
%\end{frame}
%------------------------------------------------
%\begin{frame}
%\frametitle{Bullet Points}
%\begin{itemize}
%\item Lorem ipsum 45 \(45\) \textbf{45} \(\textbf{45}\) dolor sit amet, consectetur adipiscing elit
%\item Aliquam blandit faucibus nisi, sit amet dapibus enim tempus eu
%\item Nulla commodo, erat quis gravida posuere, elit lacus lobortis est, quis porttitor odio mauris at libero
%\item Nam cursus est eget velit posuere pellentesque
%\item Vestibulum faucibus velit a augue condimentum quis convallis nulla gravida
%\end{itemize}
%\end{frame}


%\begin{frame}
%\frametitle{Multiple Columns}
%\begin{columns}[c] % The "c" option specifies centered vertical alignment while the "t" option is used for top vertical alignment

%\column{.45\textwidth} % Left column and width
%\textbf{Heading}
%\begin{enumerate}
%\item Statement
%\item Explanation
%\item Example
%\end{enumerate}

%\column{.5\textwidth} % Right column and width
%Lorem ipsum dolor sit amet, consectetur adipiscing elit. Integer lectus nisl, ultricies in feugiat rutrum, porttitor sit amet augue. Aliquam ut tortor mauris. Sed volutpat ante purus, quis accumsan dolor.

%\end{columns}
%\end{frame}

%----------------------------------------------------------------------------------------

\end{document} 
