%%%%%%%%%%%%%%%%%%%%%%%%%%%%%%%%%%%%%%%%%
% Beamer Presentation
% LaTeX Template
% Version 1.0 (10/11/12)
%
% This template has been downloaded from:
% http://www.LaTeXTemplates.com
%
% License:
% CC BY-NC-SA 3.0 (http://creativecommons.org/licenses/by-nc-sa/3.0/)
%
%%%%%%%%%%%%%%%%%%%%%%%%%%%%%%%%%%%%%%%%%

%----------------------------------------------------------------------------------------
%	PACKAGES AND THEMES
%----------------------------------------------------------------------------------------

\documentclass{beamer}

\mode<presentation> {

\usetheme{Madrid}
\usecolortheme{dolphin}
%\setbeamertemplate{footline} % To remove the footer line in all slides uncomment this line
%\setbeamertemplate{footline}[page number] % To replace the footer line in all slides with a simple slide count uncomment this line

%\setbeamertemplate{navigation symbols}{} % To remove the navigation symbols from the bottom of all slides uncomment this line
}

\usepackage{graphicx} % Allows including images
\usepackage{ragged2e} % Jusitfy
\usepackage{booktabs} % Allows the use of \toprule, \midrule and \bottomrule in tables
\usepackage{lmodern}
\usepackage{amsfonts}
\usepackage{amsmath}
\usepackage{mathtools}
\usepackage{multicol}
\usepackage{listings} % C++ code
\lstset{language=C++,
                basicstyle=\footnotesize\ttfamily,
                keywordstyle=\footnotesize\color{blue}\ttfamily,
}
\addtobeamertemplate{block begin}{}{\justifying} %Justify
%----------------------------------------------------------------------------------------
%	TITLE PAGE
%----------------------------------------------------------------------------------------

\title[Disjoint Sets]{Disjoint Sets} % The short title appears at the bottom of every slide, the full title is only on the title page

\author{Ulises M\'endez Mart\'{i}nez} % Your name
\institute[UTM] % Your institution as it will appear on the bottom of every slide, may be shorthand to save space
{
Algorist Weekly Talks \\ % Your institution for the title page
\medskip
\textit{ulisesmdzmtz@gmail.com} % Your email address
}
\date{\today} % Date, can be changed to a custom date

\begin{document}

\begin{frame}
\titlepage % Print the title page as the first slide
\end{frame}


%--------------------------------------------------------
%-- OVERVIEW SECTION
%--------------------------------------------------------
\begin{frame}
\frametitle{Overview} % Table of contents slide, comment this block out to remove it
\tableofcontents % Throughout your presentation, if you choose to use 
\end{frame}

%--------------------------------------------------------
%	PRESENTATION SLIDES
%--------------------------------------------------------

%--------------------------------------------------------
%   PROBLEM MOTIVATION
%--------------------------------------------------------
\section{Motivation} 

\subsection{New Year Permutation}
\begin{frame}
\frametitle{B. New Year Permutation}

\begin{block}{Codeforces Round 500, Problem B}
User ainta has a permutation $p_1, p_2, \dots, p_n$. As the New Year is coming, he wants to make his permutation as pretty as possible.

Permutation $a_1, a_2,  \dots, a_n$ is prettier than permutation $b_1, b_2, \dots, b_n$, if and only if there exists an integer $k$ ($1 \le k \le n$) where $a_1=b_1$,$a_2=b_2$, ..., $a_{k-1}=b_{k-1}$ and $a_k < b_k$ all holds.

As known, permutation $p$ is so sensitive that it could be only modified by swapping two distinct elements. But swapping two elements is harder than you think. Given an $nxn$ binary matrix $A$, user ainta can swap the values of $p_i$ and $p_j$ ($1 \le i,j \le n, i \ne j$) if and only if $A_{i,j}=1$.

Given the permutation $p$ and the matrix $A$, user ainta wants to know the prettiest permutation that he can obtain.
\end{block}

\end{frame}
\begin{frame}
\frametitle{Examples}

\begin{block}{Input}
\begin{multicols}{2}
$7$\\
$5\ 2\ 4\ 3\ 6\ 7\ 1$\\
$0001001$\\
$0000000$\\
$0000010$\\
$1000001$\\
$0000000$\\
$0010000$\\
$1001000$\\
$5$\\
$4\ 2\ 1\ 5\ 3$\\
$00100$\\
$00011$\\
$10010$\\
$01101$\\
$01010$
\newline
\\
\end{multicols}
\end{block}

\begin{block}{Output}
\begin{multicols}{2}
$1\ 2\ 4\ 3\ 6\ 7\ 5$ \\
$1\ 2\ 3\ 4\ 5$
\end{multicols}

\end{block}

\end{frame}
%--------------------------------------------------------

\section{Disjoint Sets}
\subsection{Union-Find Key Concepts}
\begin{frame}
\frametitle{Union Find Theory}
\begin{itemize}
	\item Representative
	\item Disjoint Sets
	\item Path compression
	\item Union by Rank
\end{itemize}
\end{frame}
%--------------------------------------------------------
%   CODE IMPLEMENTATION
%--------------------------------------------------------
\section{Implementation}
\begin{frame}[fragile]
\frametitle{ Implementation part I }
\begin{example}[ C++ Implementation ]
\begin{lstlisting}

#define MAX 1000000
int p[MAX], rank[MAX];

int find_set(int x){
   if (x != p[x])
      p[x] = find_set(p[x]);
   return p[x];
}

void union_set(int x, int y){
   link(find_set(x), find_set(y));
}
\end{lstlisting}
\end{example}
\end{frame}

\begin{frame}[fragile]
\frametitle{ Implementation part II }
\begin{example}[ C++ Implementation ]
\begin{lstlisting}

void make_set(int x){
   p[x] = x;
   rank[x] = 0;
}

void link(int x, int y){
   if (rank[x] > rank[y])
      p[y] = x;
   else{
      p[x] = y;
      if (rank[x] == rank[y])
         rank[y] = rank[y] + 1;
   }
}
\end{lstlisting}
\end{example}
\end{frame}

%--------------------------------------------------------
%   REFERENCES
%--------------------------------------------------------
\begin{frame}
\frametitle{ References }
\begin{itemize}
		\item Argentina Training Camp 2012: https://goo.gl/iesQFX
	\item Wikipedia: https://goo.gl/3TuSCl
	\item TopCoder Tutorial: https://goo.gl/O8pi6C
	\item Hackerearth Notes: https://goo.gl/TmDEYH
	\item Visualgo site: http://visualgo.net/ufds
\end{itemize}
\end{frame}

\end{document} 
